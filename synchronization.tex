\section{Self-Suspending Tasks in Multiprocessor Synchronizations}
\label{sec:syn}

In this section, we consider the analysis of self-suspensions that arise on multiprocessors under \emph{partitioned fixed-priority (\pfp)} scheduling when tasks synchronize access to shared resources (\eg, shared I/O devices, communication buffers, or scheduler locks) with suspension-based locks (\eg, binary semaphores). Unfortunately, some of the misconceptions surrounding the analysis of self-suspensions on uniprocessors also spread to the analysis of partitioned multiprocessor real-time locking protocols. In particular, as we show with a counterexample, the analysis framework to account for the additional interference due to \emph{remote blocking} first introduced by \citet{lakshmanan-2009}, and reused in several other works~\cite{zeng-2011,bbb-2013,yang-2013,kim-2014,han-2014,carminati-2014,yang-2014},  is flawed. Further, we identify a second related problem in an analysis by \citet{NBN:11}. Finally, a straightforward solution for these problems is discussed. 

\subsection{Existing analysis strategies}
\label{sec:papers}

\pfp scheduling is a widespread choice in practice due to the wide support in industrial standards such as AUTOSAR, and in many RTOSs like VxWorks, RTEMS, ThreadX, \etc Under \pfp scheduling, each task has a fixed base priority and is statically assigned to a specific processor, and each processor is scheduled independently as a uniprocessor.

\iffalse
At runtime, a resource-holding job may be assigned a temporarily elevated effective priority according to the employed locking protocol. Thus on each processor, the ready job with highest effective priority is scheduled.\todo{Why are effective priorities relevant here? Just cut it?} 
\fi

Under partitioned scheduling, a resource accessed by tasks from different processors is called a \emph{global resource}, otherwise it is called a \emph{local resource}. When a job requests a global resource, it may incur \emph{remote blocking} if the global resource is held by a job on another processor. Also, a job may incur \emph{local blocking} if it is prevented from being scheduled by a resource-holding job of a lower-priority task on its local processor. 

Under suspension-based protocols, such as the \emph{multiprocessor priority ceiling protocol (MPCP)}~\cite{rajkumar-1990}, tasks that are denied to access global shared resources are suspended. From the perspective of the local schedule on each processor, remote blocking, caused by external events (\ie, resource contention due to tasks on other, independently scheduled processors), pushes the execution of higher-priority tasks to a later time point regardless of the schedule on the local processor (\ie, even if the local processor is idle), thus may cause additional interference on lower-priority tasks. As a result, remote blocking must be considered as a self-suspension in the analysis. In contrast, local blocking takes place only if a local lower-priority task holds the resource (\ie, if the local processor is busy). Consequently, like in the uniprocessor case, local blocking is accounted for as regular blocking, and not as a self-suspension.

A safe yet pessimistic strategy, called \emph{suspension-oblivious analysis}, is to model remote blocking as computation. By overestimating the processor demand of self-suspending, higher-priority tasks, the additional delay due to deferred execution is \emph{implicitly} accounted for as part of regular interference analysis. \citet{block-2007} first used this strategy in the context of partitioned and global \emph{earliest deadline first (EDF)} scheduling;  \citet{lakshmanan-2009} also adopted this approach in their analysis of ``virtual spinning,'' where tasks suspend when blocked on a lock, but at most one task per processor may compete for a global lock at any time. However, while suspension-oblivious analysis is conceptually straightforward, it can also pessimistically overestimate response times by a factor linear in both the number of tasks and the ratio of the largest and the shortest periods~\cite{wieder-2013}.


A less pessimistic alternative to suspension-oblivious analysis is to \emph{explicitly} bound the effects of deferred execution due to remote blocking. Following this approach, \citet{lakshmanan-2009} proposed the following response-time analysis framework that takes into account the amount of remote blocking to bound the worst case interference.

In \equref{wcrt} below, let $B_k^r$ denote an upper bound on the maximum remote blocking that a job of $\tau_k$ incurs, and let $\fun{hp(k)}$ and $\fun{lp(k)}$ denote the tasks with higher and lower priority than $\tau_k$, respectively. Furthermore, $P(\tau_k)$ denotes the tasks that are assigned on the same processor as $\tau_k$, and $s_k$ is the maximum number of critical sections of $\tau_k$, and $C_{l,j}^{\prime}$ is an upper bound on the execution time of the $j$\xth critical section of $\tau_l$. \citet{lakshmanan-2009} claimed that, if $R_k^{n+1} = R_k^n \leq D_k$ for some $n > 0$, where $R_k^0 = C_k + B_k^r$ and
\begin{align}
\label{eq:wcrt}
R_k^{n+1} = C_k +  B_k^l + B_k^r + \sum_{\tau_i \in \fun{hp(k)} \cap P(\tau_k)} \left \lceil \frac{R_k^n + B_i^r}{T_i} \right \rceil \cdot C_i + s_k \cdot \sum_{\tau_l \in \fun{lp(k)} \cap P(\tau_k)} \max_{1 \leq j < s_l} C_{l,j}^{\prime},
\end{align}
then task $\tau_k$ is schedulable and its response time is bounded by $R_k^n$. This  response-time analysis framework~\cite{lakshmanan-2009} was subsequently reused to analyze (several variants of) the \mpcp \cite{yang-2013,kim-2014,carminati-2014,yang-2014}  and to compare  different locking protocols under \pfp scheduling~\cite{zeng-2011,bbb-2013,han-2014}.

In \equref{wcrt}, the additional interference on $\tau_k$ due to the lock-induced deferred execution of higher-priority tasks is captured by the term ``$+ B^r_i$'' in the interference bound  $\left \lceil \frac{R_k^n + B_i^r}{T_i} \right \rceil \cdot C_i$. Unfortunately, this fails to guarantee a safe response-time bound in certain corner cases, as can be demonstrated with the following counterexample.




\subsection{A counterexample}
\label{sec:counterexample}

We show the existence of a schedule in which a task that is considered schedulable according to the analysis in \cite{lakshmanan-2009} is in fact unschedulable.

\input{counterexample}
\begin{table} 
\centering
\caption{Task parameters}
\begin{tabular}{|c|c|c|c|c|c|c|} \hline
$\tau_k$ & $C_k$ & $T_k$ ($= D_k$) & $N_{k,1}$ & $L_{k,1}$ & $N_{k,2}$ & $L_{k,2}$\\ \hline
$\tau_1$ & 2           & 6  & 0 & 0 & 0 & 0\\ \hline
$\tau_2$ & $4+2\epsilon$ & 13 & 1 & $\epsilon$ & 0 & 0\\ \hline
$\tau_3$ & $\epsilon$    & 14 & 0 & 0 & 1 & $\epsilon$\\ \hline
$\tau_4$ & 6             & 14 & 1 & $4-\epsilon$ & 1 & $\epsilon$\\ \hline
\end{tabular}
\label{tab:parameters}
\end{table}

%\multicolumn{2}{c}{} 
%\multirow{2}{*}{Progress Mechanism}


Consider four implicit deadline sporadic tasks ${\tau_1, \tau_2, \tau_3, \tau_4}$ (with parameters listed in \tabref{parameters}, where $N_{k,1}$ ($N_{k,2}$) denotes the maximum number of requests that a job of $\tau_k$ can issue to resource $\res_1$ ($\res_2$), and $L_{k,1}$ ($L_{k,2}$) denotes the corresponding maximum critical section length), ordered by decreasing order of priority, that are scheduled on two processors using \pfp scheduling. $\tau_1$, $\tau_2$ and $\tau_3$ are assigned to processor 1, while $T_4$ is assigned to processor 2. Jobs of $\tau_2$ and $\tau_4$   each once access a shared resource $\res_1$  ($N_{2,1} = 1$ and $N_{4,1} = 1$). Jobs of $\tau_4$ each once access a shared resource $\res_2$ ($N_{4,2} = 1$). Each job of $\tau_2$ uses $\res_1$ for a duration of at most $L_{2,1} = \varepsilon < 1$ time units (an arbitrarily small quantity), and each job of $\tau_4$ uses $\res_1$ and $\res_2$ for at most $L_{4,1} = 4-\varepsilon$ and $L_{4,1} = \varepsilon$ time units respectively. 

Consider the response-time of $\tau_3$. Since $\tau_3$ is the lowest-priority task on its processor, it does not incur any local blocking (\ie,$s_3 \sum_{\tau_l \in \fun{lp}(3) \cap P(\tau_3)} \max_{1 \leq j < s_l} C_{l,j}^{\prime} = 0$). While each time $\tau_3$ requests $\res_2$, it may be delayed by $\tau_4$ for at most $\epsilon$. Thus, the maximum remote blocking of $\tau_3$ is bounded by $B_3^r = \epsilon$ \footnote{In general, the upper bound on blocking of course depends on the specific locking protocol in use, but in this example, by construction, the stated bound holds under any reasonable locking protocol. Recent surveys of multiprocessor semaphore protocols may be found in \cite{bbb-2013,yang-2015}.}. With regard to the remote blocking incurred by each higher-priority task, we have $B_1^r = 0$ because $\tau_1$ does not request any global resource. Further, each time when a job of $T_2$ requests $\res_1$, it may be delayed for a duration of at most $4-\varepsilon$, thus $B_2^r = 4-\varepsilon$. Therefore, according to \equref{wcrt}, we have
\begin{align*}
& R_3^0 = \varepsilon + \varepsilon = 2\varepsilon, \\
& R_3^1 = 2\varepsilon + 0 + \left \lceil \frac{2\varepsilon + 0}{6} \right \rceil \cdot 2 + \left \lceil \frac{2\varepsilon + 4 - \varepsilon}{13} \right \rceil \cdot (4+2\varepsilon) = 2\varepsilon + 1 \cdot 2 + 1 \cdot (4+2\varepsilon) = 6+4\varepsilon, \\
& R_3^2 = 2\varepsilon + 0 + \left \lceil \frac{6+4\varepsilon + 0}{6} \right \rceil \cdot 2 + \left \lceil \frac{6+4\varepsilon + 4-\varepsilon}{13} \right \rceil \cdot (4+2\varepsilon) = 2\varepsilon + 2 \cdot 2 + 1 \cdot (4+2\varepsilon) = 8+4\varepsilon, \\
& R_3^3 = 2\varepsilon + 0 + \left \lceil \frac{8+4\varepsilon + 0}{6} \right \rceil \cdot 2 + \left \lceil \frac{8+4\varepsilon + 4-\varepsilon}{13} \right \rceil \cdot (4+2\varepsilon) = 2\varepsilon + 2 \cdot 2 + 1 \cdot (4+2\varepsilon) = 8+4\varepsilon.
\end{align*}
 
As a result, $R_3 = 8+4\varepsilon < 14 = D_3$, and $\tau_3$ is considered to be schedulable according to the analysis in \cite{lakshmanan-2009}. However, there exists a schedule, shown in \figref{counterexample}, where $\tau_3$  actually misses a deadline at time~20, which implies that the analysis framework in \cite{lakshmanan-2009} is too optimistic (in certain cases). 

\subsection{Incorrect Time Request Analysis With Global Resource Sharing}

Besides the over-optimistic problem found in \citep{lakshmanan-2009}, a straightforward adoption of the traditional time request analysis with global resource sharing is also not safe, as we show next.

 In \cite{NBN:11}, an \emph{interface-based analysis} based on the time request analysis was derived for real-time open systems, where each processor contains a task set and the tasks on different processors may share global resources. Intuitively, the interface abstracts the requirements of global resource sharing on each processor that should be satisfied to guarantee the schedulability of the system in the processor. In particular, the maximum blocking time that a task $\tau_k$ can tolerate without missing its deadline, denoted by $\fun{mtbt}_k$, is evaluated as following. 

Starting from the schedulability condition, $\tau_k$ is considered schedulable in a single processor if

\begin{align}\label{eq:rbf-1}
\exists t \in (0,D_k]: \fun{rbf_{FP}}(k,t) \leq t, 
\end{align}
where $\fun{rbf_{FP}}(k,t)$ is the \emph{request bound function} of $\tau_k$, which is computed by

\begin{align}\label{eq:rbf-2}
\fun{rbf_{FP}} = C_k + B_k + \sum_{\tau_i \in \fun{hp}(k)} \left \lceil \frac{t}{T_i} \right \rceil \cdot C_i.
\end{align}

In \equref{rbf-2}, $B_k$ denotes the total blocking time of $\tau_k$. By substituting $B_k$ and $\fun{mtbt}_k$, $\fun{mtbt}_k$ is then computed by

\begin{align}\label{eq:bloc-tolerate}
\fun{mtbt}_k = \max_{0<t \leq D_k} \left( t - ( C_k + \sum_{\tau_i \in \fun{hp}(k)} \left \lceil \frac{t}{T_i} \right \rceil \cdot C_i ) \right).
\end{align}

However, from to the analysis in \secref{counterexample}, we can already infer that the analysis used in \equref{rbf-1} and \equref{rbf-2}, which ignores the effects of remote blocking on interference and quantifies the total interference from a task $\tau_i$ on $\tau_k$ over a duration of length $t$ as $\left \lceil \frac{t}{T_i} \right \rceil \cdot C_i$, is over optimistic. 

Consider $\tau_3$ in the previous example (with parameters listed in \tabref{parameters}). According to \equref{bloc-tolerate}, $\fun{mtbt}_3$ is $12 - (\epsilon + \lceil 12 / 6 \rceil \cdot 2 + \lceil 12 / 13 \rceil \cdot (4+2\epsilon)) = 4-3\epsilon$ (when $t=12$), which implies that $\tau_3$ can tolerate a maximum blocking time of at least $4-3\epsilon$ without missing its deadline. However, this is not true since $\tau_3$ is unschedulable even without incurring any blocking, as shown in \figref{counterexample}.

\subsection{A Safe Response Time Bound}
\label{sec:safe_bound}

In \equref{wcrt}, the remote blocking of each higher-priority task (\ie, $B_i^r$) is counted in a similar way as release jitter. However, it is not sufficient to count the duration of remote blocking as release jitter (already explained in section 3.2.3). A straightforward fix is thus to replace $B_i^r$, in the ceiling function (\ie, the second term in \equref{wcrt}), with a larger value such as $D_i$ (as proved/discussed in section 3.3) or $R_i - C_i$ (as proved / discussed in section 3.3). Similarly, replacing $\sum_{\tau_i \in \fun{hp}(k)} \lceil t / T_i \rceil \cdot C_i$ in \equref{rbf-2} and \equref{bloc-tolerate} with $\sum_{\tau_i \in \fun{hp}(k)} \lceil (t+D_i) / T_i \rceil \cdot C_i$ or $\sum_{\tau_i \in \fun{hp}(k)} \lceil (t+R_i-C_i) / T_i \rceil \cdot C_i$ may fix the over-optimistic problem in \cite{NBN:11}.

Further, since most papers reviewed in \secref{papers} merely reused the over-optimistic analysis framework in \cite{lakshmanan-2009}, the stated fix may be used to correct the response-time tests in these papers.