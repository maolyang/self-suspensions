\section{Self-Suspending Tasks in Multiprocessor Synchronizations}
\label{sec:syn}

In this section, we will review the general analysis strategies for tasks that synchronize access to shared resources (\eg, shared I/O devices, communication buffers, or scheduler locks) with suspension-based locks (\eg, binary semaphores) in partitioned-scheduled multiprocessors. Unfortunately, some of the misconceptions found in existing self-suspension analysis also tackled the problem of multiprocessor real-time locking protocol analysis. In particular, we will show with a counterexample that the analysis framework, first introduced in \cite{lakshmanan-2009} and reused in several other work~\cite{zeng-2011,bbb-2013,yang-2013,kim-2014,han-2014,carminati-2014,yang-2014}, to quantify the additional interference due to \emph{remote blocking} is flawed. Finally, we show a straightforward solution to fix such problem. 

\subsection{Existing analysis strategies}
\label{sec:papers}

In multiprocessor real-time systems, the \emph{partitioned fixed-priority (\pfp)} scheduling is a widespread choice in practice due to the wide support in industrial standards like POSIX and AUTOSAR, and in many RTOSs like VxWorks, RTEMS, ThreadX, \etc Under \pfp scheduling, each task has a fixed base priority and is statically assigned to a specific processor, and the tasks on each processors are scheduled as in uniprocessors. At runtime, a resource-holding job may be assigned a temporarily elevated effective priority according to the employed locking protocol. Thus on each processor, the ready job with highest effective priority is scheduled. 

Under partitioned scheduling, a resource accessed by tasks from different processors is called a \emph{global resource}, otherwise it is called a \emph{local resource}. When a job requests a global resource, it may incur \emph{remote blocking} (\ie, self-suspension) if the global resource is held by a job on another processor. Also, a job may incur \emph{local blocking} if it is prevented from being scheduled by a resource-holding job of a lower-priority task on the same processor. From the perspective of local schedule on each processor, remote blocking, caused by external events (\ie, depends on the schedule on the other processors), pushes the execution of higher-priority tasks to a later time point, thus can cause additional interference on lower-priority tasks. 

A safe yet pessimistic strategy is to convert remote blocking into computation. Accordingly, the remote blocking incurred by each higher-priority task is counted as part of interference. This strategy was used in \cite{block-2007} for partitioned \emph{earliest deadline first (EDF)} scheduling, and in \cite{lakshmanan-2009} when suspension was modeled as virtual spinning. An alternative is to bound the effects of deferred execution due to remote blocking. In the original utilization-based analysis for multiprocessor locking protocols~\cite{rajkumar-1990,RSL:88}, the additional interference from a task $\tau_i$ on task $\tau_k$ was counted as part of blocking for $\tau_k$, and was bounded by the maximum remote blocking or the WCET of $\tau_i$ (whichever is smaller). Recently, Lakshmanan \etal~\cite{lakshmanan-2009} proposed an response-time analysis framework that takes into account the amount of remote blocking to bound the worst case interference as following. 

\begin{align}
\label{eq:wcrt}
R_k^{n+1} = C_k + B_k^r + B_k^l + \sum_{\tau_i \in \fun{hp(k)} \cap P(\tau_k)} \left \lceil \frac{R_k^n + B_i^r}{T_i} \right \rceil \cdot C_i.  
\end{align}
where, $R_k^0 = C_k + B_k^r$, and $\tau_k$ is schedulable if $R_k^{n+1} = R_k^n < D_k$. 

In \equref{wcrt}, $B_k^r$ ($B_k^l$) is an upper bound on the maximum remote (local) blocking that a job of $\tau_k$ incurs, $\fun{hp(k)}$ denotes the tasks with higher priority than $\tau_k$, and $P(\tau_k)$ denotes the tasks that are assigned on the same processor as $\tau_k$. The additional interference on $\tau_k$ due to the remote blocking incurred by each higher-priority task is captured in the analysis according to the forth term in \equref{wcrt}.

Under this analysis framework, $\left \lceil \frac{R_k + B_i^r}{T_i} \right \rceil \cdot C_i$ is used as an upper bound on the total interference of $\tau_i$ on $\tau_k$. This analysis approach was reused in \cite{yang-2013,kim-2014,carminati-2014,yang-2014} to analyze the \emph{multiprocessor priority ceiling protocol (\mpcp)} and the variations of the \mpcp. The response-time analysis framework (\ie, \equref{wcrt}) was also reused in \cite{zeng-2011,bbb-2013,han-2014} to compare the schedulability performances between different locking protocols under \pfp scheduling. 

Unfortunately, the analysis approach based on \equref{wcrt} failed to guarantee a safe response time bound, for which we provide a counterexample in the following.

\subsection{A counterexample}

We show the existence of a schedule in which a task that is considered schedulable according to the analysis in \cite{lakshmanan-2009} is in fact unschedulable.

\input{counterexample}
\begin{table} 
\centering
\caption{Task parameters}
\begin{tabular}{|c|c|c|c|c|c|c|} \hline
$\tau_k$ & $C_k$ & $T_k$ ($= D_k$) & $N_{k,1}$ & $L_{k,1}$ & $N_{k,2}$ & $L_{k,2}$\\ \hline
$\tau_1$ & 2           & 6  & 0 & 0 & 0 & 0\\ \hline
$\tau_2$ & $4+2\epsilon$ & 13 & 1 & $\epsilon$ & 0 & 0\\ \hline
$\tau_3$ & $\epsilon$    & 14 & 0 & 0 & 1 & $\epsilon$\\ \hline
$\tau_4$ & 6             & 14 & 1 & $4-\epsilon$ & 1 & $\epsilon$\\ \hline
\end{tabular}
\label{tab:parameters}
\end{table}

%\multicolumn{2}{c}{} 
%\multirow{2}{*}{Progress Mechanism}


Consider four implicit deadline sporadic tasks ${\tau_1, \tau_2, \tau_3, \tau_4}$ (with parameters listed in \tabref{parameters}), ordered by decreasing order of priority, are scheduled on two processors using \pfp scheduling. $\tau_1$, $\tau_2$ and $\tau_3$ were assigned on processor 1, while $T_4$ was assigned on processor 2. Each job of $\tau_2$ and each job of $\tau_4$ will respectively access a shared resource $\res_1$ at most once before completion ($N_{2,1} = 1$ and $N_{4,1} = 1$). Each job of $\tau_2$ uses $\res_1$ for a time of length at most $L_{2,1} = \varepsilon$ ( an arbitrarily small quantity), and each job of $\tau_4$ uses $\res_1$ for at most $L_{4,1} = 4-\varepsilon$. 

Consider the response-time of $\tau_3$. Since $\tau_3$ is the lowest-priority task on its processor and $\tau_3$ does not request any resource, it does not incur any blocking (\ie, $B_3^l = B_3^l = 0$). With regard to the remote blocking incurred by each higher-priority task, we have $B_1^r = 0$ because $\tau_1$ does not request any global resource. Further, each time when a job of $T_2$ requests $\res_1$, it may be delayed (\ie, self-suspend) for a duration of at most $4-\varepsilon$ before $T_4$ releases $\res_1$. Thus, the maximum remote blocking of $\tau_2$ is bounded by $B_2^r = 4-\varepsilon$ \footnote{The upper bound of blocking time of course depends on the locking protocol in use, but in this example, the stated bound holds under any reasonable locking protocols. A recent survey of multiprocessor semaphore protocols may be found in \cite{bbb-2013,yang-2015}}. Therefore, according to \equref{wcrt}, we have 
\begin{align*}
& R_3^1 = \varepsilon + 0 = \varepsilon, \\
& R_3^2 = \varepsilon + 0 + 0 + \left \lceil \frac{\varepsilon + 0}{6} \right \rceil \cdot 2 + \left \lceil \frac{\varepsilon + 4 - \varepsilon}{13} \right \rceil \cdot (4+2\varepsilon) = \varepsilon + 1 \cdot 2 + 1 \cdot (4+2\varepsilon) = 6+3\varepsilon, \\
& R_3^3 = \varepsilon + 0 + 0 + \left \lceil \frac{6+3\varepsilon + 0}{6} \right \rceil \cdot 2 + \left \lceil \frac{6+3\varepsilon + 4-\varepsilon}{13} \right \rceil \cdot (4+2\varepsilon) = \varepsilon + 2 \cdot 2 + 1 \cdot (4+2\varepsilon) = 8+3\varepsilon, \\
& R_3^4 = \varepsilon + 0 + 0 + \left \lceil \frac{8+3\varepsilon + 0}{6} \right \rceil \cdot 2 + \left \lceil \frac{8+3\varepsilon + 4-\varepsilon}{13} \right \rceil \cdot (4+2\varepsilon) = \varepsilon + 2 \cdot 2 + 1 \cdot (4+2\varepsilon) = 8+3\varepsilon.
\end{align*}
 
As a result, $R_3 = 8+3\varepsilon < 14 = D_3$, and $\tau_3$ is considered to be schedulable according to the analysis in \cite{lakshmanan-2009}. However, there exists a schedule as shown in \figref{counterexample} where $\tau_3$ is actually unschedulable, which implies that the analysis framework in \cite{lakshmanan-2009} is too optimistic. 

Based on the above discussion, we can infer that the analysis used in \cite{NBN:11}, which ignored the effects of remote blocking on interference and quantified the total interference from a task $\tau_i$ on $\tau_k$ over a duration of length $t$ as $\left \lceil \frac{t}{T_i} \right \rceil \cdot C_i$, was also over optimistic. 

\subsection{A Safe Response Time Bound}
\label{sec:safe_bound}

In \equref{wcrt}, the remote blocking of each higher-priority task (\ie, $B_i^r$) is counted in a similar way as release jitter. However, it is not sufficient to count the duration of remote blocking as release jitter (explained in XXX (\emph{this will be explained in previous sectioins})). A straightforward fix is thus to replace $B_i^r$ with a larger value for each higher-priority task (\ie, $\tau_i$) when $B_i^r > 0$.

\begin{lemma}
\label{lem:new_framework}
The worst-case response time of a task $\tau_k$ is upper bounded by the smallest solution of 
\begin{align}
\label{eq:fix}
R_k = C_k + B_k^r + B_k^l + \sum_{\tau_i \in \fun{hp}(k) \cap P(\tau_k) \wedge B_i^r > 0} \left \lceil \frac{R_k + R_i - C_i}{T_i} \right \rceil \cdot C_i
+ \sum_{\tau_j \in \fun{hp}(k) \cap P(\tau_k) \wedge B_j^r = 0} \left \lceil \frac{R_k}{T_j} \right \rceil \cdot C_j.
\end{align}
\end{lemma}

\begin{figure}[!ht]
\captionsetup{belowskip=-1pt}
\begin{center}
\includegraphics[width=11cm]{max_interference}
\caption{Release pattern for $\tau_i$.}
\label{fig:max_interference}
\end{center}
\end{figure}


\begin{proof}
At any point in time while a job of $\tau_k$ is released but not finishes, it is either \textbf{(i)} executing, \textbf{(ii)} blocked due to tasks on the other processors, \textbf{(iii)} blocked due to tasks on its local processor, or \textbf{(iv)} preempted by higher-priority tasks on its local processor. 
By definition, $C_k$, $B_k^r$, and $B_k^l$ bound the durations of \textbf{(i)}, \textbf{(ii)}, and \textbf{(iii)}, respectively.

For any higher-priority local task, \ie, $\forall \tau_h \in \fun{hp}(k) \cap P(\tau_k)$, the interference of $\tau_h$ on a job of $\tau_k$ is maximized if the carry-in job executes as late as possible and the following jobs execute as soon as possible, as shown in \figref{max_interference}. Correspondingly, the interference of $\tau_h$ on $\tau_k$ is bounded by 
\begin{align*}
I_{h,k}^1 = \left \lceil \frac{R_k + R_h - C_h}{T_h} \right \rceil \cdot C_h.
\end{align*}

For any higher-priority local task $\tau_j$ (\ie, $\forall \tau_j \in \fun{hp}(k) \cap P(\tau_k)$) that does not incur remote blocking (\ie, $B_j^r = 0$), $\tau_j$ will not suspend while executing (\ie, does not self-suspend). According to the critical instant theory~\cite{LL-1973}, a job of $\tau_k$ incurs maximum interference by $\tau_j$ when the job releases simultaneously with the release of $\tau_j$. Therefore, according to the standard request demand analysis~\cite{audsley-1993}, the maximum interference from $\tau_j$ on $\tau_k$ is given by 
\begin{align*}
I_{j,k}^2 = \left \lceil \frac{R_k}{T_j} \right \rceil \cdot C_j.
\end{align*}

Consequently, the total interference on any job of $\tau_k$ (\ie, the duration of \textbf{(iv)}) is bounded by summing up $I_{i,k}^1$ for each higher-priority local task $\tau_i$ if $B_i^r > 0$, and $I_{j,k}^2$ for each higher-priority local task $\tau_j$ if $B_j^r = 0$, which is given in the fourth and the fifth terms in \equref{fix}.
\end{proof} 

It is noted that, \lemref{new_framework} can also be used to fix the over-optimistic problem in \cite{NBN:11}. Further, since most papers reviewed in \secref{papers} merely reused the over-optimistic analysis framework in \cite{lakshmanan-2009}, an analysis based on \lemref{new_framework} may be used to fix the response-time tests in these papers.